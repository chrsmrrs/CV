\documentclass[11pt, a4paper, DIV=12]{scrartcl}

\usepackage[english]{babel}
\usepackage{cmap}       
\usepackage[T1]{fontenc}
\usepackage[utf8]{inputenc}
\usepackage{exscale}
\usepackage[l2tabu, orthodox]{nag}
\usepackage[symbol]{footmisc}
\renewcommand{\thefootnote}{\fnsymbol{footnote}}

\setlength\parindent{0in} 
\usepackage[usenames,dvipsnames]{xcolor} 
\definecolor{maroon}{rgb}{0.5, 0.0, 0.0}
\definecolor{upmaroon}{RGB}{133, 75, 91}
\definecolor{upgrey}{RGB}{99, 124, 128}
\definecolor{upblack}{RGB}{64, 62, 48}
\definecolor{upbeige}{RGB}{224, 229, 224}

\usepackage[
bookmarks, 
colorlinks, 
breaklinks, 
pdftitle={Christopher Morris -- CV},
pdfauthor={Christopher Morris}
]{hyperref}
\hypersetup{linkcolor=black,citecolor=blue,filecolor=black,urlcolor=black}

\usepackage{url}

\usepackage{microtype}
\usepackage{ellipsis}
\usepackage{xspace}
\usepackage{hfoldsty}

\usepackage{ifthen}
\newcommand{\CC}[1][]{$\text{C\hspace{-.25ex}}^{_{_{_{++}}}}
	\ifthenelse{\equal{#1}{}}{}{\text{\hspace{-.625ex}#1}}$}

%\addtokomafont{disposition}{\rmfamily}
%\addtokomafont{descriptionlabel}{\rmfamily}
%\usepackage{lmodern}
\usepackage[varqu]{zi4}
%\usepackage[sfdefault]{classico}
\usepackage[tt=false]{libertine}
%\usepackage[]{classico}
%\usepackage[varqu]{zi4}
\usepackage[libertine]{newtxmath}
%\usepackage[]{helvet}

\begin{document}

\section*{\textcolor{upmaroon}{Curriculum vitæ}}
\vspace{-20pt}
\hrulefill
\subsection*{Address}
\noindent
\begin{tabular}{l}
Christopher Morris\\
Theaterstraße 35-39\\ 
Office 215\\ 
52062 Aachen, Germany\\
E-mail: \texttt{{morris@cs.rwth-aachen.de}} \\
Homepage: \url{http://www.christophermorris.info}\\
\end{tabular}

\subsection*{Areas of Specialization}
\noindent
\begin{tabular}{p{15.0cm}}
Machine learning for graphs (graph neural networks, equivariant neural networks, graph kernels) from a theoretical as well as applied viewpoint, and its application in combinatorial optimization and the life sciences.
\end{tabular}

\subsection*{Education}
\noindent
\begin{tabular}{p{3.0cm}p{11.5cm}}
	06/2015--12/2019&PhD in Computer Science, TU Dortmund University, Germany, final grade: 1.0 with highest distinction  (best possible grade)\\
	01/2018--04/2018& Research stay at Stanford, Infolab (Jure Leskovec)\\
	04/2012--05/2015&MSc in Computer Science, TU Dortmund University, Germany, final grade: 1.0 (best possible grade)\\
	10/2008--12/2011&BSc in Computer Science, TU Dortmund University, Germany\\
	08/1998--06/2007&University Entrance Qualification, Erzbisch\"ofliches St.-Angela-Gymnasium, Wipperf\"urth\\
\end{tabular}

\subsection*{Employment}
\begin{tabular}{p{3.0cm}p{11.5cm}}
    06/2022--present& Tenure-track assistant professor (``junior professorship'' (W1)) at RWTH Aachen, Department of Computer Science\\
	06/2021--06/2022& Postdoctoral researcher at McGill University, Department of Computer Science, in the group of Siamak Ravanbakhsh (DAAD IFI scholarship) and member of the Mila -- Quebec AI Institute, Montréal\\
	03/2020--05/2021& Postdoctoral researcher in the group of Andrea Lodi, Department of Mathematical and Industrial Engineering, Polytechnique Montréal\\
	06/2015--12/2019&PhD student and research associate, TU Dortmund University, Department of Computer Science, within the DFG Collaborative Research Center SFB 876, advised by Petra Mutzel (now University of Bonn) and Kristian Kersting (Technical University of Darmstadt)\\
	08/2007--04/2008&Mandatory civil service (German Red Cross)\\
\end{tabular}

\renewcommand{\refname}{\large\bfseries Publications}



\begin{thebibliography}{5}

\subsubsection*{Conference Papers}

	\bibitem{Mor+2015}
	Christopher Morris, Gaurav Rattan, Sandra Kiefer, Siamak Ravanbakhsh.
	\emph{SpeqNets: Sparsity-aware permutation-equivariant graph networks},
	International Conference on Machine Learning (ICML), 2022, spotlight presentation.

	\bibitem{Mor+2015}
	Elias B.\, Khalil, Christopher Morris, Andrea Lodi.
	\emph{MIP-GNN: A data-driven framework for guiding combinatorial solvers}, 
	 AAAI Conference on Artificial Intelligence (AAAI), 2022, oral presentation.
	
	\bibitem{Mor+2015}
    	Leonardo Cotta, Christopher Morris, Bruno Ribeiro,
	\emph{Reconstruction for Powerful Graph Representations},
	Neural Information Processing Systems (NeurIPS), 2021.	

	\bibitem{Mor+2015}
	Quentin Cappart, Didier Chételat, Elias Khalil, Andrea Lodi, Christopher Morris, Petar Veli\v{c}kovi\'{c}.\footnote[1]{Alphabetical author order.}
	\emph{Combinatorial optimization and reasoning with graph neural networks},
    	International Joint Conference on Artificial Intelligence (IJCAI), 2021, pages 4348--4355.

	\bibitem{Mor+2015}
	Christopher Morris, Matthias Fey, Nils M.~Kriege.
	\emph{The Power of the Weisfeiler-Leman Algorithm for Machine Learning with Graphs},
	International Joint Conference on Artificial Intelligence (IJCAI), 2021, pages 4543--4550.

	\bibitem{Mor+2020}
   	 Christopher Morris, Gaurav Rattan, Petra Mutzel.
	\emph{Weisfeiler and Leman Go Sparse: Towards Scalable Higher-order Graph Neural Networks},
	Neural Information Processing Systems (NeurIPS), pages 21824--21840, 2020.

	\bibitem{Fey+2020}
	Matthias Fey, Jan E. Lenssen, Christopher Morris, Jonathan Masci, Nils M. Kriege.
    \emph{Deep Graph Matching Consensus},
	International Conference on Learning Representations (ICLR), 2020.
	
	\bibitem{Oet+2020}
	Lutz Oettershagen, Nils Kriege, Christopher Morris, Petra Mutzel.
	\emph{Temporal Graph Kernels for Classifying Dissemination Processes},
	SIAM International Conference on Data Mining (SDM), pages 496--504, 2020.
	
	\bibitem{Mor+2019}
	Christopher Morris, Martin Ritzert, Matthias Fey, William L. Hamilton, Jan Eric Lenssen, Gaurav Rattan, Martin Grohe.
	\newblock \emph{Weisfeiler and Leman Go Neural: Higher-order Graph Neural Networks},
	\newblock AAAI Conference on Artificial Intelligence (AAAI), pages 4602--4609, 2019.
	
	\bibitem{Yin+2018}
	Rex Ying, Jiaxuan You, Christopher Morris, Xiang Ren, William L. Hamilton, Jure Leskovec.
	\emph{Hierarchical Graph Representation Learning with Differentiable Pooling},
	Neural Information Processing Systems (NeurIPS), pages 4805--4815, 2018, spotlight presentation.
	
	\bibitem{Kri+2018}
	Nils M.~Kriege, Christopher Morris, Anja Rey, Christian Sohler.$^*$
	\emph{A Property Testing Framework for the Theoretical Expressivity of Graph Kernels},
	International Joint Conference on Artificial Intelligence (IJCAI), pages 2348--2354, 2018.
	
	\bibitem{Mor+2017}
    Christopher Morris, Kristian Kersting, Petra Mutzel.
	\emph{Glocalized Weisfeiler-Lehman Graph Kernels: Global-Local Feature Maps of Graphs},
	IEEE International Conference on Data Mining (ICDM), pages 327--336, 2017, full paper.
	
	\bibitem{Mor+2017}
	Christopher Morris, Nils M.~Kriege.
	\emph{Recent Advances in Kernel-Based Graph Classification},
	European Conference on Machine Learning \& Principles and Practice of Knowledge Discovery in Databases (ECML PKDD), pages 388--392, 2017.
	
	\bibitem{Mor+2016}
    Christopher Morris, Nils M.~Kriege, Kristian Kersting, Petra Mutzel.
	\emph{Faster Kernels for Graphs with Continuous Attributes via Hashing},
	IEEE International Conference on Data Mining (ICDM), pages 1095--1100, 2016.
\subsubsection*{Journal Articles}

\bibitem{Mor+2020}
Lutz Oettershagen, Nils M.~Kriege, Christopher Morris, and Petra Mutzel.
\emph{Classifying Dissemination Processes in Temporal Graphs},
Big Data 8 (5), pages 363--378, 2020.

\bibitem{Mor+2020}
Nils M.~Kriege,  Fredrik D. Johansson, Christopher Morris.$^*$
\emph{A Survey on Graph Kernels},
Applied Network Science 5 (1), pages 1-42, 2020.

\bibitem{Mor+2020}
Nils M.~Kriege,  Marion Neumann, Christopher Morris, Kristian Kersting, Petra Mutzel.
\emph{A Unifying View of Explicit and Implicit Feature Maps for Structured Data: Systematic Studies of Graph Kernels},
Data Mining and Knowledge Discovery 33 (6), pages 1505-1547, 2019.

\bibitem{Mor+2016}
Fritz B\"okler, Mathias Ehrgott, Christopher Morris, Petra Mutzel.$^*$
\emph{Output-sensitive Complexity of Multiobjective Combinatorial Optimization},
Journal of Multicriteria Decision Analysis 24 (1-2), pages 25-36, 2017.

\subsubsection*{Book Chapters}

\bibitem{Mor+2015}
Christopher Morris.
\emph{Graph Neural Networks: Graph Classification},
in \emph{Graph Neural Networks: Foundations, Frontiers, and Applications}, edited by Peng Cui, Jian Pei, Lingfei Wu, Liang Zhao, pages 179-193, Springer, 2021.

\bibitem{Mor+2020}
Christopher Morris.
\emph{Lernen mit Graphen: Kern- und neuronale Methoden}, in 
\emph{Ausgezeichnete Informatikdissertationen 2019}, edited by Steffen H{\"o}lldobler, Sven Apel, Felix Freiling, Hans-Peter Lehnhof, Gustaf Neumann, R{\"u}diger Reischuk, Kai U. R{\"o}mer, Bj{\"o}rn Scheuermann, Nicole Schweikardt, Myra Spiliopoulou, Sabine S{\"u}sstrunk, Klaus Wehrle, LNI, D-19, Gesellschaft f{\"u}r Informatik (GI), 2020. 

\subsubsection*{Workshop Papers (Peer-reviewed)}

\bibitem{Mor+2015}
Maxime Gasse, Quentin Cappart, Jonas Charfreitag, Laurent Charlin, Didier Chételat, Antonia Chmiela, Justin Dumouchelle, Ambros Gleixner, Aleksandr M. Kazachkov, Elias Khalil, Pawel Lichocki, Andrea Lodi, Miles Lubin, Chris J. Maddison, Christopher Morris, Dimitri J. Papageorgiou, Augustin Parjadis, Sebastian Pokutta, Antoine Prouvost, Lara Scavuzzo, Giulia Zarpellon, Linxin Yang, Sha Lai, Akang Wang, Xiaodong Luo, Xiang Zhou, Haohan Huang, Shengcheng Shao, Yuanming Zhu, Dong Zhang, Tao Quan, Zixuan Cao, Yang Xu, Zhewei Huang, Shuchang Zhou, Chen Binbin, He Minggui, Hao Hao, Zhang Zhiyu, An Zhiwu, Mao Kun.
\emph{The Machine Learning for Combinatorial Optimization Competition (ML4CO): Results and Insights}, NeurIPS 2021 Competitions and Demonstrations Track, PMLR 176:220-231, 2022.

\bibitem{Mor+2015}
Christopher Morris, Gaurav Rattan, Sandra Kiefer, Siamak Ravanbakhsh.
\emph{SpeqNets: Sparsity-aware permutation-equivariant graph networks},
Geometrical and Topological Representation Learning (GT-RL, ICLR 2022), also accepted at ICML 2022.

\bibitem{Mor+2020}
Christopher Morris, Gaurav Rattan, Petra Mutzel.
\emph{Weisfeiler and Leman go sparse: Towards scalable higher-order graph embeddings},
Graph Representation Learning and Beyond (GRL+, ICML 2020), also accepted at NeurIPS 2020.

\bibitem{Mor+2020}
Christopher Morris, Nils M. Kriege, Franka Bause, Kristian Kersting, Petra Mutzel, Marion Neumann.
\emph{TUDataset: A collection of benchmark datasets for learning with graphs},
Graph Representation Learning and Beyond (GRL+, ICML 2020).

\bibitem{Yin+2018}
Rex Ying, Jiaxuan You, Christopher Morris, Xiang Ren, William L. Hamilton, Jure Leskovec.
\emph{Hierarchical Graph Representation Learning with Differentiable Pooling},
KDD Deep Learning Day 2018, also accepted at NeurIPS 2018.

\subsubsection*{Thesis}

\bibitem{Mor+2019}
Christopher Morris.
\emph{Learning with Graphs: Kernel and Neural Approaches}, PhD thesis, TU Dortmund University, 2019.

\bibitem{Mor+2015}
Christopher Morris.
\emph{Enumeration Complexity of Multicriteria Linear Optimization}, MSc thesis, TU Dortmund University, 2015.
	
\subsubsection*{Submitted Papers}

\bibitem{Mor+2015}
Chendi Qian, Gaurav Rattan, Floris Geerts, Christopher Morris, Mathias Niepert.
\emph{Ordered Subgraph Aggregation Networks}, (Preprint: \texttt{arxiv:abs/2206.11168}), 2022. 

\bibitem{Mor+2015}
Christopher Morris, Yaron Lipman, Haggai Maron, Bastian Rieck, Nils M. Kriege, Martin Grohe, Matthias Fey, Karsten Borgwardt.
\emph{Weisfeiler and Leman go Machine Learning: The Story so far}, (Preprint: \texttt{arxiv:abs/2112.09992}), 2021. 

\bibitem{Mor+2015}
Quentin Cappart, Didier Chételat, Elias Khalil, Andrea Lodi, Christopher Morris, Petar Veli\v{c}kovi\'{c}.$^*$
\emph{Combinatorial optimization and reasoning with graph neural networks},
CoRR, (Preprint: \texttt{arxiv:abs/2102.09544}), 2021. Submitted to Journal of Machine Learning Research, extend version of [3].
\end{thebibliography}

\subsection*{Academic Honors}
\begin{tabular}{p{2.1cm}p{12.0cm}}
	2021& DFG Emmy Noether fellow\\
	2020& Dissertation award of TU Dortmund University (awarded for best Computer Science PhD thesis in 2020)\\
	2020& Nominated (by TU Dortmund University) for the dissertation award of the German computer science association (GI Dissertationspreis)\\
\end{tabular}

\subsection*{Invited Talks}
\begin{tabular}{p{2.1cm}p{12.0cm}}

11/2022& ELLIIT Hybrid AI Workshop, Linköping University, \emph{Graph Neural Networks for Data-driven Optimization}\\
09/2022& Mini-symposium on ``Advances in Learning for Graphs, Manifolds, and Geometric Data'' at the SIAM Conference on Mathematics of Data Science (MDS22), San Diego, \emph{Learning with Graphs: Graph Neural Networks and the Weisfeiler-Leman algorithm}\\	
07/2022& Banff International Research Station for Mathematical Innovation and Discovery (BIRS) workshop ``Deep Exploration of non-Euclidean Data with Geometric and Topological Representation Learning'',  \emph{Towards Understanding the Expressive Power of Graph Networks}\\	
02/2022& University of Windsor (virtual), \emph{Learning with Graphs: From Theory to Applications}\\
01/2022&RWTH Aachen University (virtual), \emph{Learning with Graphs: From Theory to Applications}\\
12/2021&RWTH Aachen University (virtual), \emph{Learning with Graphs: From Theory to Applications}\\
\end{tabular}

\begin{tabular}{p{2.1cm}p{12.0cm}}
10/2021&University of Oxford (virtual), \emph{Learning with Graphs: Graph Neural Networks and the Weisfeiler-Leman algorithm}\\
07/2021&Saarland University (virtual), \emph{Machine Learning with Graphs:
	From Theory to Applications in Science and Engineering}\\
07/2021&University of Hannover (virtual), \emph{Machine Learning with Graphs:
	From Theory to Applications in Science and Engineering}\\
11/2020&McGill University (virtual), \emph{Limits of Graph Neural Networks and the Weisfeiler-Leman algorithm}\\
11/2020&INFORMS Annual Meeting (virtual), \emph{Limits Of Graphs Neural Networks For Combinatorial Optimization} \\
10/2019&IBM Research, Zürich, \emph{Graph Classification: Kernel and Neural Approaches}\\
05/2019&NEC Research, Heidelberg, \emph{Graph Classification: Kernel and Neural Approaches}\\ 
03/2018&Stanford University, SNAP, Infolab, \emph{Learning Higher-order Graph Embeddings: Theory and Practice}\\
07/2017&RWTH Aachen University, Chair of Logic and the Theory of Discrete Systems, \emph{Graph Classification: Kernels and Beyond}\\
\end{tabular}


\subsection*{Teaching}
\begin{tabular}{l}
Supervised eight bachelor and master thesis, one intern. \\[0.5em]
\end{tabular}

\begin{tabular}{p{2.1cm}p{12.0cm}}
03/2022	& Lecture \emph{Introduction to Graph Neural Networks: Machine Learning with Graphs} in the \emph{Dataninja Spring School} organized by the University of Bielefeld\\
11/2021 & Guest lecture \emph{Introduction to Graph Neural Networks} in the \emph{Applied Machine Learning} class, McGill University\\	
SS 2019&Proseminar \emph{Graph Algorithms} (supervised students and helped with organization)\\
SS 2018&Seminar \emph{Algorithm Engineering} (supervised students and helped with organization)\\
SS 2017&Seminar \emph{Algorithm Engineering} (supervised students and helped with organization)\\
\end{tabular}

\begin{tabular}{p{2.1cm}p{12.0cm}}
WS 2016/17&Student project group \emph{Algorithm Engineering for Graph Data Mining} (co-organizer), Seminar \emph{Algorithms Unplugged} (supervised students and helped with organization)\\
SS 2016& Seminar \emph{Algorithm Engineering}, Seminar \emph{Graph Mining} (supervised students and helped with organization)\\
As a student&Programming tutorials for engineering students, teaching assistant for a course on theoretical computer science\\
\end{tabular}

\subsection*{Service to the Profession}
\begin{tabular}{p{14.5cm}}	
	Panellist at the \emph{Workshop on
		Geometrical and Topological Representation Learning} (ICLR 2022 workshop) \\[0.5em]
\end{tabular}

\begin{tabular}{p{14.5cm}}
	Co-organizer of the graph machine learning reading group at Mila -- Quebec AI Institute (\url{grlmila.github.io}) \\[0.5em]	
\end{tabular}

\begin{tabular}{p{14.5cm}}
Co-organizer of the NeurIPS 2021 competition \emph{Machine Learning for Combinatorial Optimization} (\url{www.ecole.ai/2021/ml4co-competition}) \\[0.5em]	
\end{tabular}

\begin{tabular}{p{14.5cm}}
Co-organizer of the Dagstuhl seminar \emph{Graph Embeddings: Theory Meets Practice} (March 27–30 2022, Dagstuhl Seminar 22132, together with Martin Grohe (RWTH Aachen University), Stephan Günnemann (TU Munich), Stefanie Jegelka (MIT)) \\[0.5em]
\end{tabular}

\begin{tabular}{p{14.5cm}}
Initiator of \url{www.graphlearning.io}, a large collection of benchmark datasets for graph classification and regression\\[0.5em]
\end{tabular}

\begin{tabular}{p{14.5cm}}	
Area chair for LoG Conference 2022 (Learning on Graphs Conference), Senior program committee member AAAI 2023\\[0.5em]
\end{tabular}

\begin{tabular}{p{14.5cm}}	
Program committee member for IJCAI 2019, NeurIPS  2019, AAAI 2020, ICML 2020, IJCAI 2020, ECML-PKDD 2020, NeurIPS 2020, ICLR 2021, AAAI 2021, ICML 2021, IJCAI 2021, NeurIPS 2021, ICLR 2022, ICML 2022, NeurIPS 2022 Competition Track, NeurIPS 2022, ICLR 2023 \\[0.5em]
\end{tabular}

\begin{tabular}{p{14.5cm}}
Program committee member for \emph{Representation Learning on Graphs and Manifolds} (ICLR 2019 workshop), \emph{Learning and Reasoning with Graph-Structured Data} (ICML 2019 workshop), \emph{Graph Representation Learning} (NeurIPS 2019 workshop), \emph{Graph Representation Learning and Beyond} (ICML 2020 workshop), \emph{Graphs and more Complex Structures for Learning and Reasoning} (AAAI 2021 workshop), \emph{Workshop on Graph Learning Benchmarks} (The Web Conference 2021 workshop),  \emph{Graphs and more Complex Structures for Learning and Reasoning} (AAAI 2022 workshop), GroundedML: Workshop on Anchoring Machine Learning in Classical Algorithmic Theory (ICLR 2022 workshop), \emph{Workshop on Graph Learning Benchmarks} (The Web Conference 2022 workshop), \emph{Mining and
Learning with Graphs} (ECML 2022 workshop)\\[0.5em]
\end{tabular}

\begin{tabular}{p{14.5cm}}
(Sub-)Reviewer for WALCOM 2017, ISAAC 2018, ALENEX 2019, ESA 2018, ICALP 2020 \\[0.5em]
\end{tabular}

\begin{tabular}{p{14.5cm}}
Occasional reviews for Transactions on Machine Learning Research (2$\times$2022), IEEE Transactions on Pattern Analysis and Machine Intelligence (2$\times$2020), Journal of Machine Learning Research (2020, 2021), Bioinformatics (2022), IEEE Transactions on Knowledge and Data Engineering (2021), INFORMS Journal on Computation (2022), INFORMS Journal on Optimization (2021), ACM Transactions on Knowledge Discovery from Data (2019), IEEE Transactions on Cybernetics (2020), IEEE Transactions on Mobile Computing (2020), Elsevier Signal Processing (2021)\\[0.5em]
\end{tabular}

\begin{tabular}{p{14.5cm}}
Member of the appointment commission for the professorship \emph{Data Mining} (TU Dortmund University, 2017)
\end{tabular}

\subsection*{Grants}
\begin{tabular}{p{14.5cm}}
	DFG Emmy Noether grant (own funding 970\,460 + 776 460\,€)\\[0.5em]
	
	RWTH Junior Principal Investigator Fellowship (own funding 958\,918\,€)\\[0.5em]

	DAAD IFI postdoc scholarship for a 13 month stay at the Mila--Quebec AI Institute (own funding 38\,909\,€)\\[0.5em]
		
	Research associate and PhD student  within the Collaborative Research Center SFB 876, assisted in preparing the  grant proposal for project A6 \emph{Resource-efficient Graph Mining}
\end{tabular}


\subsection*{Other}

\begin{tabular}{l}
	\textsf{\textbf{\em Computational Skills}} Python, C\hspace{-1pt}+\hspace{-1pt}+, \LaTeX, NumPy, Scikit-learn, PyTorch, PyTorch Geometric\\
	\textsf{\textbf{\em Languages}} German (native), English (fluent)\\
	\textsf{\textbf{\em Citizenship}} German and British
\end{tabular}





\subsection*{References}

\begin{tabular}{l}
	Professor Petra Mutzel (main PhD supervisor)\\
	Computational Analytics,\\ 
	Department of Computer Science,\\
	University of Bonn \\
	\href{mailto:petra.mutzel@cs.uni-bonn.de}{\texttt{petra.mutzel@cs.uni-bonn.de}}\\
\end{tabular}\\[0.5em]


\begin{tabular}{l}
	Professor Martin Grohe\\
	Logic and Theory of Discrete Systems,\\ 
	Department of Computer Science,\\
	RWTH Aachen University \\
	\href{mailto:grohe@informatik.rwth-aachen.de}{\texttt{grohe@informatik.rwth-aachen.de}}\\
\end{tabular}\\[0.5em]

\begin{tabular}{l}
	Professor Kristian Kersting\\ 
	Machine Learning Group,\\
	Department of Computer Science,\\
	TU Darmstadt\\
	\href{mailto:kersting@cs.tu-darmstadt.de}{\texttt{kersting@cs.tu-darmstadt.de}}\\
\end{tabular}\\[0.5em]

\begin{tabular}{l}
	Professor Andrea Lodi\\
	Andrew H. and Ann R. Tisch Professor,\\
	Jacobs Technion-Cornell Institute,\\
	Cornell University\\
	\href{mailto:andrea.lodi@cornell.edu}{\texttt{andrea.lodi@cornell.edu}}\\
\end{tabular}\\[0.5em]



\vfill{} 
\begin{center}
{\scriptsize Last updated: \today}
\end{center}



\end{document}
